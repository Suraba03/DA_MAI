\documentclass[12pt]{article}
\usepackage{fullpage}
\usepackage{multicol,multirow}
\usepackage{tabularx}
\usepackage{ulem}
\usepackage[utf8]{inputenc}
\usepackage[russian]{babel}

\usepackage{pgfplots}

\begin{document}
    
    \section*{Лабораторная работа №\,4 по курсу дискрeтного анализа: Поиск 
    образца в строке}

Выполнил студент группы 8О-207Б-20 МАИ \textit{Чекменёв Вячеслав Алексеевич}.

    \subsection*{Условие}

    \begin{enumerate}
        \item \textbf{Упрощённый вариант:} поиск одного образца в тексте с 
        помощью алгоритма Z-блоков.
        \item \textbf{Алфавит:} строчные латинские буквы.
    \end{enumerate}

    \subsection*{Метод решения}
    Объявим две функции - zfunc() и findentries(), в первой вычислим z функцию для строки вида <pattern>@<text>. Во второй пройдемся по строке и найдем индексы, в которых z функция равна длин паттерна.

    \subsection*{Дневник отладки}
    \begin{enumerate}
        \item Сначала написал программу с простой z функцией, которая работает за квадрат
        \item Потом переписал, как на лекции
    \end{enumerate}

    \subsection*{Тест производительности}
    Сравнение производительности производилось с наивным алгоритмом поиска 
    подстроки. Время выполнения в миллисекундах.
\\

    \begin{tabular}{ | c | c | c | }
        \hline
            Кол-во строк & Z-блоки & Наивный алгоритм \\ \hline
            1000 & 856 & 1697 \\
            10000 & 8670 & 14975 \\
            100000 & 80367 & 199760 \\
        \hline
    \end{tabular}
    
    \\
    
Исходя из результатов можно выяснить, что алгоритм работает значительно быстрее наивного алгоритма поиска, так как сложность нашего алгоритма O(n+k)

    \subsection*{Выводы}
    После выполнения данной ЛР я познакомился с простейшим алгоритмом поиска образца в строке, который работает за линию.\\
    Написание действительно оказалось простым, так как ввод был стандартный в отличие от вариантов не на оценку 3.\\
    Плюсы данного алгоритма - сложность (асимптотическая = O(n+k), где k - длина образца, n - длина текста) на уровне с другими (более сложными) алгоритмами, легкость написания.\\
    Минусы - не так быстр как, например, алгоритм Бойера-Мура, так как не использует мелкие ускроения по типо правил хорошего суффикса и плохого символа.
    

\end{document}
