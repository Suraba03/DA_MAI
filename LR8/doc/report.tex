\documentclass[12pt]{article}
\usepackage{fullpage}
\usepackage{multicol,multirow}
\usepackage{tabularx}
\usepackage{ulem}
\usepackage[utf8]{inputenc}
\usepackage[russian]{babel}
\usepackage{pgfplots}


\begin{document}

    \section*{Лабораторная работа №\,8 по курсу дискрeтного анализа: 
    Жадные алгоритмы}

    Выполнил студент группы М8О-307Б-20 МАИ \textit{Чекменев Вячеслав}.

    \subsection*{Условие}
 
    \begin{enumerate}
    \item Разрабтать жадный алгоритм решения задачи, определяемой своим
    вариантом. Доказать его корректность, оценить скорость и объём
    затрачиваемой оперативной памяти.
    \item \textbf{Вариант 3: Максимальный треугольник.} Заданы длины N отрезков, необходимо выбрать три таких отрезка,
    которые образовывали бы треугольник с максимальной площадью.
    Формат входных данных: на первой строке находится число N, за
    которым следует N строк с целыми числами-длинами отрезков. Формат
    выходных данных: если никакого треугольника из заданных отрезков
    составить нельзя — 0, в противном случае на первой строке площадь
    треугольника с тремя знаками после запятой, на второй строке — длины
    трёх отрезков, составляющих этот треугольник. Длины должны быть
    отсортированы.
    \end{enumerate}

    \subsection*{Метод решения}

        Жадный алгоритм заключается в принятии локально оптимальных решений на каждом этапе, допуская, что конечное решение также окажется оптимальным, поэтому к этой задаче можно применить жадный 
    алгоритм, поскольку чтобы её решить, мы должны на каждом шаге брать наибольшие стороны из списка, начиная с самой большой, чтобы в будущем получить самую большую площадь.
    
    Идея решения в том, чтобы отсортировать вектор сторон, идти по нему циклом так, чтобы брать всегда наибольшую тройку чисел и считать их площадь, тогда нам не придется перебирать все тройки и алгоритм будет работать в среднем за линейное время.

    \subsection*{Описание программы}

    Программа состоит из одного файла.

    \subsection*{Дневник отладки}

    \begin{enumerate}
    \item WA9 -- ошибка: останавливался и  выводил ответ, когда находил первую положительную площадь
    \item Исправил: проходил полностью по вектору и ОКнул
    \end{enumerate}


    \subsection*{Производительность}

    Временная сложность алгоритма — $O(nlogn)$, так как была использована сортировка вектора ($O(nlogn)$), заполнение вектора ($O(n)$) и проход по нему ($O(n)$)

    \subsection*{Недочёты}

    Похоже, что их нет либо я не нашел.

    \subsection*{Выводы}

    Проделав лабораторную работу, я ознакомился с тем, как применять концепцию жадных алгоритмов. Применил формулу Герона из школьной геометрии, доказал истинность жадного алгоритма.

\end{document}
