\documentclass[12pt]{article}

\usepackage{fullpage}
\usepackage{multicol,multirow}
\usepackage{tabularx}
\usepackage{ulem}
\usepackage[utf8]{inputenc}
\usepackage[russian]{babel}

\usepackage{pgfplots}
% \pgfplotsset{compat=1.9}


% Оригиналный шаблон: http://k806.ru/dalabs/da-report-template-2012.tex

\begin{document}

\section*{Лабораторная работа №\,2 по курсу дискрeтного анализа: Сбаласированные деревья}

Выполнил студент группы 08-207 МАИ \textit{Чекменев Вячеслав Алексеевич}.

\subsection*{Условие}

% Кратко описывается задача: 
\begin{enumerate}
\item Необходимо создать программную библиотеку, реализующую указанную структуру данных, 
на основе которой разработать программу-словарь. В словаре каждому ключу, представляющему 
из себя регистронезависимую последовательность букв английского алфавита длиной не более 
256 символов, поставлен в соответствие некоторый номер, от $0$ до $2^{64} - 1$. Разным словам 
может быть поставлен в соответствие один и тот же номер.
\item Вариант задания: Патриция.
\item Операции: вставка, поиск, удаление элемента, сохранение словаря в файл, загрузка словаря 
из файла.
\end{enumerate}

\subsection*{Метод решения}

Из прошлого опыта я решил выделить для узла и дерева отдельные классы с независимой структурой для лучшей модульности.

\subsection*{Описание программы}

Программа состоит из:
\begin{itemize}
    \item двух заголовочных файлов, один принадлежит классу самого дерева, второй структуре узла дерева 
    \item двух файлов с описанием функций для класса и структуры
    \item и клиента
\end{itemize}

Основые методы дерева:
\begin{itemize}
    \item $ ReturnNode() $ -- идея состоит в том, чтобы спускаться по дереву вниз, пока бит не станет меньше либо равен биту с прошлого рекурсивного вызова, тогда останавливаем рекурсию и сравниваем наше значение со значением в нвнешнем узле. Если они равны, то узел найден, иначе - нет.
    \item $ AddNode() $ -- делаем то же, что и в $ ReturnNode() $, если узел не найден, то ищем элемент еще раз, но с условием, что бит (различия) должен только увеличиваться, тогда между узлом, в который придем и его отцом вставим элемент.
    \item $ DeleteNode() $ -- здесь будет сложно уместить, но идея в том, что находим элемент Bnode, который ссылается на наш элемент Anode снизу, тогда есть всего три случая:
    \begin{enumerate}
        \item Bnode лист -- все его ссылки ведут наверх
        \item Bnode == Anode
        \item Bnode -- не лист, самый сложный случай, но надо по сути переназначить ссылки, что работает за константу
    \end{enumerate}
\end{itemize}


\subsection*{Дневник отладки}

\begin{enumerate}
    \item 17.04 -- начал писать поиск и вставку исключительно по лекциям, не подглядывая в алгоритм
    \item 19.04 -- ничего почти не работает, решил посмотреть статью на медиуме по патриции
    \item 23.04 -- написал удаление и вставку, однако использовал указатель на отца, что выглядит как плохой код
    \item 24.04 -- исправил указатель на отца, написав функцию по поиску отца
    \item 25.04 -- написал удаление и заслал на чекер, получил WA на 5 тесте, думаю, что это связано со вставкой
    \item 27.04 -- нашел ошибку во вставке, неправильно распределял указатели прошел 5 тест, получил ML на 6 тесте
    \item 29.04 -- переписал предствление ключа в структуре так чтобы хранить не последовательность из 0 и 1, а слово, записанное латиницей, получил WA на 7 тесте, думаю, что это связано с удалением
    \item 01.05 -- сделал так, что бит отличия будет отсчитываться от правого конца слова, однако пройти 7 тест это не помогло
    \item 03.05 -- нашел ошибку в удалении, неправильно было прописано удаление при случае 2, исправил, прошел тест 7, получил WA на 9, это уже файлы
    \item 05.05 -- начал писать файлы, идея была в том, чтобы просто вставлять в новое дерево при загрузке
    \item 06.05 -- получил RE на 12 тесте, подумал, что это связано с идеей загрузки через вставку, переписал так, чтобы вставки не было, а узлы просто создавались в памяти, однако получил RE на 9 тесте
    \item 06.05 -- устал перекидывать код в один файл, чтобы отослать на чекер, решил воспользоваться make, однако постоянно получал CE из-за Time Limit
    \item 08.05 -- спросил у преподавателей и 10 одногруппников, в итоге мы нашли ошибку, она была в том, что я по привычке подключил библиотеку bits/stdc++.h, которая отнимает много времени для компиляции, подключил все библиотеки отдельно, заработало
    \item 10.05 -- вернулся к прошлой идее и нашел две ошибки -- 
    \begin{enumerate}
        \item Cсылался на NULL в одном месте
        \item Не проверял файл на пустоту при загрузке
    \end{enumerate}
        исправил, получил ОК.
\end{enumerate}
Здесь я не изложил все ошибки, так как было много мелких чисто по невнимательности.

\subsection*{Тест производительности}

Во всех графиках по оси Y отложено время выполнения (в миллисекундах), по оси X — количество 
команд.

\subsubsection*{Вставка}

\begin{tikzpicture}
    \begin{axis} [
        legend pos = north west,
        ymin = 0
    ]
    \legend{
        Patricia,
        std::map
    };
    \addplot coordinates {
        (1000,30) (10000,140) (100000,1400) (1000000,17040)
    };
    \addplot coordinates {
        (1000,160) (10000,1390) (100000,15703)
    };
    \end{axis}
\end{tikzpicture}



\begin{tabular}{ | 1 | 1 | 1 | }
    \hline
        Кол-во строк & std::map & Patricia \\ \hline
        1000 & 160 & 30 \\
        10000 & 1390 & 140 \\
        100000 & 15703 & 1400 \\
        1000000 & ----- & 17040 \\
        
    \hline
\end{tabular}

\subsubsection*{Поиск}

\begin{tikzpicture}
    \begin{axis} [
        legend pos = north west,
        ymin = 0,
    ]
    \legend{
        Patricia,
        std::map
    };
    \addplot coordinates {
        (1000,6) (10000,30) (100000,350) (1000000,7000)
    };
    \addplot coordinates {
        (1000,158) (10000,1449) (100000,15218)
    };
    \end{axis}
\end{tikzpicture}

\begin{tabular}{ | 1 | 1 | 1 | }
    \hline
        Кол-во строк & std::map & Patricia \\ \hline
        1000 & 158 & 6 \\
        10000 & 1449 & 30 \\
        100000 & 15218 & 350 \\
        1000000 & ----- & 7000 \\
    \hline
\end{tabular}

\subsubsection*{Удаление}

\begin{tikzpicture}
    \begin{axis} [
        legend pos = north west,
        ymin = 0
    ]
    \legend{
        Patricia,
        std::map
    };
    \addplot coordinates {
        (1000,10) (10000,80) (100000,900)  (1000000,11500)
    };
    \addplot coordinates {
        (1000,146) (10000,1561) (100000,15832)
    };
    \end{axis}
\end{tikzpicture}

\begin{tabular}{ | 1 | 1 | 1 | }
    \hline
        Кол-во строк & std::map & Patricia \\ \hline
        1000 & 146 & 10 \\
        10000 & 1561 & 80 \\
        100000 & 15832 & 900 \\
        1000000 & ----- & 11500 \\
    \hline
\end{tabular}       



Таким образом, операции над Patricia имеют ту же сложность, что и над std::map, по времени работает быстрее чем std::map, однако я думаю, что это из-за специфики рандомных данных, так как патриция работает по маске. Поэтому больше узлов больше занимают память, однако это повышает скорость.


\subsection*{Недочёты}

В коде, который получил ОК на чекере некоторые функцие делают немного больше, чем нужно, например функция поиска всегда считает бит различия, однако это нужно только один раз при вставке. А также написано много других ненужных функций чисто для отладки, таких как Traverse(), Print()... 


\subsection*{Выводы}

    Эта лабораторная работа оказалась самой сложной за два курса, я никогда так долго не работал над одной задачей, могу выделить основые моменты, которые, я думаю, окажутся полезными в будущем:
    \begin{enumerate}
        \item не писать грязный код, на каждом шаге очищать его от ненужного мусора, который в будущем может помещать и вызвать баги
        \item работа с отладчиком, я использовал gdb, узнал много новых фичей в нем, которые помогли найти много segfault-ов
        \item распределение программы на несколько частей, не только физическое разделение на файлы, но и логическое 
        \item прорабатывать одну идею до конца, не переключаться на другие, пока не зайду в тупик
        \item использовать контроль версий git: я использовал github, туда заливал код в новую ветку и, если считал, что этот чем-то сильно лучше того, что лежит в мастер ветке, делал merge, так мне была видее весь прогресс
    \end{enumerate}
    В общем данная ЛР выдалась сложной, но интересной и полезной.

\end{document}
