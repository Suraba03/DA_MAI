\documentclass[12pt]{article}
\usepackage{fullpage}
\usepackage{multicol,multirow}
\usepackage{tabularx}
\usepackage{ulem}
\usepackage[utf8]{inputenc}
\usepackage[russian]{babel}
\usepackage{pgfplots}


\begin{document}

    \section*{Лабораторная работа №\,7 по курсу дискрeтного анализа: 
    Динамическое программирование}

    Выполнил студент группы М8О-307Б-20 МАИ \textit{Чекменев Вячеслав}.

    \subsection*{Условие}
 
     \item При помощи метода динамического программирования разработать алгоритм 
    решения задачи, определяемой своим вариантом; оценить время выполнения 
    алгоритма и объем затрачиваемой оперативной памяти. Перед выполнением 
    задания необходимо обосновать применимость метода динамического 
    программирования.
    \item \textbf{Вариант 3: Количество чисел}.
    Задано целое число n. Необходимо найти количество натуральных (без нуля) чисел, которые меньше n по
значению и меньше n лексикографически (если сравнивать два числа как строки), а так же делятся на m без
остатка.

    \subsection*{Метод решения}
Будем на каждой итерации цикла применять формулу $T(n,m) = n/m - 10^{len_n-1}/m$ для подсчета делителей которые лексикографически меньше n. Динамическое программирование основано на переходе для 1 состояния $dp[n][m]$, для второго $dp[n/10][m]$ и так далее, пока n не равно нулю. 
Для задачи $n, m$, где $len_n$ -- количество разрядов числа $n$ подзадача будет состоять из количества чисел, удовлетворяющих формуле выше и по длине равных length(n).


    \subsection*{Описание программы}

    Программа состоит из одного файла.

    \subsection*{Дневник отладки}

    \begin{enumerate}
    \item Сначала делал наивным способом, он не зашел, потом подумал, что можно смотреть разряды отдельно
    \end{enumerate}


    \subsection*{Сложность}

    Сложность равна O(length(str(n)))

    \subsection*{Недочёты}

    Не выявил.

    \subsection*{Выводы}

    Проделав лабораторную работу, познакомился с новым подходом решения 
    алгоритмических задач — динамическим программированием, а также написал интересный алгоритм который работает существенно быстрее, чем наивный для данной задачи.

\end{document}
