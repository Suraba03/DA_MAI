\documentclass[12pt]{article}
\usepackage{fullpage}
\usepackage{multicol,multirow}
\usepackage{tabularx}
\usepackage{ulem}
\usepackage[utf8]{inputenc}
\usepackage[russian]{babel}
\usepackage{pgfplots}

\usepackage{listings}
\usepackage{color}

\definecolor{dkgreen}{rgb}{0,0.6,0}
\definecolor{gray}{rgb}{0.5,0.5,0.5}
\definecolor{mauve}{rgb}{0.58,0,0.82}

\lstset{frame=tb,
  language=Java,
  aboveskip=3mm,
  belowskip=3mm,
  showstringspaces=false,
  columns=flexible,
  basicstyle={\small\ttfamily},
  numbers=none,
  numberstyle=\tiny\color{gray},
  keywordstyle=\color{blue},
  commentstyle=\color{dkgreen},
  stringstyle=\color{mauve},
  breaklines=true,
  breakatwhitespace=true,
  tabsize=3
}

\begin{document}

    \section*{Курсовой проект по курсу дискрeтного анализа: 
    Аудиопоиск}

    Выполнил студент группы М8О-307Б-20 МАИ \textit{Чекменев Вячеслав}.
    \subsection*{Условие}
    \begin{itemize}
    \item Реализуйте систему для поиска аудиозаписи по небольшому отрывку.
./prog index --input <input file> \
--output <index file> --- индексация файлов
    \item ./prog search --index <index file> \
--input <input file> \
--output <output file> --- поиск файлов. Все файлы будут даны в формате MP3 с частотой дискретизации 44100Гц.
Входные файлы содержат в себе имена файлов с аудио записями по одному файлу в строке.
Результатом ответа на каждый запрос является строка с названием файла, с которым произошло
совпадение, либо строка “! NOT FOUND”, если найти совпадение не удалось.
    
    \end{itemize}
    \subsection*{Метод решения}

    \begin{enumerate}
        \item Из полученной зависимости амплитуд от частот получить частоты.
        \item Однако, в одной песне диапазон «сильных» частот может варьироваться. Поэтому, вместо того, чтобы за сразу проанализировать весь частотный диапазон, мы можем выбрать несколько более мелких интервалов. Выбор можно сделать, основываясь на частотах, которые обычно присущи важным музыкальным компонентам, и проанализировать их по отдельности. Найти среди них самые "сильные" частоты (с максимальным значением амплитуды) в каждом из интервалов между точками 0, 40, 80, 120, 180, 300, 500, 1000. Это действие проделать для каждого "окна" в 4096 отсчетов. 
        \item Тогда мы получим таблицу в строках которой будет храниться нужные нам частоты. Эти сведения и формируют сигнатуру для конкретного анализируемого блока данных, а она, в свою очередь, является частью сигнатуры всей песни.
        \item Для упрощения поиска музыкальных композиций их сигнатуры используются как ключи в хэш-таблице. Когда набор частот, для которых найдена сигнатура, появился в произведении, и идентификатор самого произведения (название песни и имя исполнителя, например). Возьмем хэш-функцию: $$h(f1, f2, ... , f7) = \sum_{k=0}^7 {(f_k - (f_k mod 2)) * 10^k}$$
        \item Тогда набор хэшей для каждого трека и будет являться сигнатурой наших аудиозаписей
        \item При флаге "index" будем просто вычислять сигнатуры и записывать их в вектор, а потом и в файл
        \item Всякий раз, когда удаётся обнаружить совпадающий хэш-тег, число возможных совпадений уменьшается, но вероятно, что только лишь эти сведения не позволят нам настолько сузить диапазон поиска, чтобы остановиться на единственной правильном треке. Поэтому в алгоритме распознавания музыкальных произведений нам нужно проверять ещё и длину временного интервала. А именно, сначала положим в 2d вектор сигнатуры исходных файлов, потом вычислим сигнатуру для сэмпла и найдем два значения в таблице сигнатур: первый элемент сигнатуры сэмпла, последний элемент сигнатуры сэмпла. А также сравним временной интервал между найденными совпадениями в сигнатурах исходных треков и длину самого сэмпла.
        \item Если матч произошел сразу с больше, чем одним треком, то выведем первый, так как тут ничего поделать нельзя.
        
    \end{enumerate}

    \subsection*{Дневник отладки}

    \begin{enumerate}
        \item 04.01.23 Изучена теория
        \item 05.01.23 Написаны тесты
        \item 10.01.23 Написан код
        \item 10.01.23 Протестирован код
    \end{enumerate}


    \subsection*{Тестирование}

    Для тестирования было взято две базы данных. Первая содержит опенинги из разных аниме (30 штук) длиной 1 минута 30 секунд и их части длиной 10 секунд, которые нам и надо найти. Вторая содержит случайно сгенерированные треки треки длиной примерно 45 секунд, а также их части длиной 15 секунд.

    \begin{itemize}
        \item Индексация треков: \\
        
            \begin{tabular}{ | l | l | }
                \hline
                    Кол-во элементов в массиве pcm & Время (в миллисекундах) \\ \hline
                    1984500 & 3914 \\
                    9922500 & 19039 \\
                    18742500 & 42447 \\
                    44982000 & 91415 \\
                    99225000 & 188870 \\
                    104958000 & 226389 \\
                \hline
            \end{tabular}
        \item Поиск: \\
        
            \begin{tabular}{ | l | l | }
                \hline
                    Кол-во элементов в массиве pcm & Время (в миллисекундах) \\ \hline
                    441000 & 819 \\
                    2205000 & 4214 \\            
                    10584000 & 20609 \\
                    16317000 & 31284 \\
                    19845000 & 37183 \\
                    55125000 & 103777 \\
                \hline
            \end{tabular}    

\\
\begin{tikzpicture}
    \begin{axis} [
        legend pos = north west,
        ymin = 0
    ]
    \legend{
        Computing signatures, 
        Searching
    };
    \addplot coordinates {
        (1984500,3914) (9922500,19039) (18742500, 42447) (44982000, 91415) (99225000, 188870) (104958000, 226389)
    };
    \addplot coordinates {
        (441000,819) (2205000,4214) (10584000, 20609) (16317000, 31284 ) (19845000, 37183) (55125000, 103777)
    };
    \end{axis}
\end{tikzpicture}
    
    Сложность алгоритмов можно оценить как O(n*log(n))
        \item Точность поиска \\ 

        \textbf{Первый тест} \\ 
        
        \begin{lstlisting}
01-Acknowledged_Product.mp3
Matches: Acknowledged_Product.mp3, 

02-Bony_Communication.mp3
Matches: Bony_Communication.mp3, 

03-Chirpy_Situation.mp3
Matches: Convivial_Measurement.mp3, Chirpy_Situation.mp3

04-Compassionate_Exam.mp3
Matches: Compassionate_Exam.mp3, 

26-Primal_Arrival.mp3
! NOT FOUND

05-Conspicuous_Bedroom.mp3
Matches: Conspicuous_Bedroom.mp3, 

06-Convivial_Measurement.mp3
Matches: Convivial_Measurement.mp3, 

07-Cozy_Communication.mp3
Matches: Cozy_Communication.mp3, 

28-Rightful_Indication.mp3
! NOT FOUND

08-Curvaceous_Elevator.mp3
Matches: Curvaceous_Elevator.mp3, 

09-Destined_Goal.mp3
Matches: Destined_Goal.mp3, 

10-Engaging_Agency.mp3
Matches: Engaging_Agency.mp3, 
        \end{lstlisting}
        Не найдено или найдено неправильно  - 3 из 12 треков => точность = 75\%

    \textbf{Второй тест} \\

        \begin{lstlisting}
01-attack_on_titan_op3.mp3
Matches: attack_on_titan_op3.mp3, 

02-attack_on_titan_op4.mp3
! NOT FOUND

03-attack_on_titan_op4_2.mp3
Matches: attack_on_titan_op4_2.mp3, 

04-death_note_op1.mp3
Matches: death_note_op1.mp3, 

05-guilty_crown_op1.mp3
Matches: guilty_crown_op1.mp3, 

06-guilty_crown_op2.mp3
Matches: guilty_crown_op2.mp3, 

11-hunterxhunter_op1.mp3
Matches: hunterxhunter_op1.mp3, 

13-naruto_storm_op4.mp3
Matches: psychopass_op4.mp3, 

14-naruto_storm_op7.mp3
Matches: naruto_storm_op7.mp3, 

15-naruto_ua_op2.mp3
Matches: naruto_ua_op2.mp3, 

17-psychopass_ed1.mp3
! NOT FOUND

18-psychopass_op2.mp3
Matches: psychopass_op2.mp3, 

19-psychopass_op4.mp3
Matches: psychopass_op4.mp3, 

20-sao_op1.mp3
Matches: sao_op1.mp3, 
        \end{lstlisting}
    Не найдено или найдено неправильно  - 3 из 14 треков => точность = 79\%

    \end{itemize}
 \\ 

    \subsection*{Недочеты}
    Возникают ошибки при сэмплах взятых из центра трека и сильно зашумленных. В таком случае сигнатуры сэмпла и трека часто отличаются.
    
    \subsection*{Выводы}

        Проделав данный курсовой проект я узнал много нового про аудиопоиск и в целом про обработку аналоговых сигналов. Глубже разобрался как работает преобразование фурье, смог достать из него не только магнитуды, но и самые сильные частоты. Также разобрался с возможными диапазонами частот для разных типов музыки. Еще мне пришлось придумать хэш функцию, которая будет создавать наименьшее количество коллизий - это у меня получилось и количество различных хэшей составляло 4/5 от всех возможных.
        
        Также столкнулся с проблемой при матчинге треков. Искать всю сигнатуру в исходной таблице сигнатур было бы неэффективно. Поэтому я воспользовался трюком: брал только перую и последнюю частоту и искал их в таблице, а также сравнивал временные интервалы в сэмпле и исходном треке.
    \end{document}
