\documentclass[12pt]{article}
\usepackage{fullpage}
\usepackage{multicol,multirow}
\usepackage{tabularx}
\usepackage{ulem}
\usepackage[utf8]{inputenc}
\usepackage[russian]{babel}
\usepackage{pgfplots}

\usepackage{listings}
\usepackage{color}

\definecolor{dkgreen}{rgb}{0,0.6,0}
\definecolor{gray}{rgb}{0.5,0.5,0.5}
\definecolor{mauve}{rgb}{0.58,0,0.82}

\lstset{frame=tb,
  language=Java,
  aboveskip=3mm,
  belowskip=3mm,
  showstringspaces=false,
  columns=flexible,
  basicstyle={\small\ttfamily},
  numbers=none,
  numberstyle=\tiny\color{gray},
  keywordstyle=\color{blue},
  commentstyle=\color{dkgreen},
  stringstyle=\color{mauve},
  breaklines=true,
  breakatwhitespace=true,
  tabsize=3
}

\begin{document}

    \section*{Курсовой проект по курсу дискрeтного анализа: 
    Быстрое преобразование Фурье}

    Выполнил студент группы М8О-307Б-20 МАИ \textit{Чекменев Вячеслав}.
    \subsection*{Условие}
 
    \begin{enumerate}
    \item Разработать программу на языке C или C++, реализующую указанный
    алгоритм. Формат входных и выходных данных описан в задании.
    \item Реализуйте алгоритм быстрое преобразование Фурье для действительного сигнала. В качестве первого аргумента вашей программе передаётся название mp3 файла который необходимо обработать. Для каждых 4096 отсчётов с шагом задаваемым вторым аргументом. Перед преобразованием Фурье необходимо подействовать на отсчёты окном Ханна.
    
    В качестве результата для каждого набора отсчётов выведите наибольшее заначение по абсолютной величине полученное после преобразования Фурье.
    \end{enumerate}

    \subsection*{Метод решения}

    \begin{enumerate}
        \item Преобразовать полученный файл mp3 в вектор pcm, то есть значений амплитуд в каждом отсчете с частотой дискретизации 44,1 kHz
        \item Из полученного вектора извлечь значения амплитуд идущие через значение, заданное вторым аргументом
        \item Далее разбить данные значения на группы по 4096 штук и для каждой применить окно
        Ханна: $$w[n] = \frac{1}{2} * (1 - \cos{\frac{2\pi n}{N - 1}}); 0 \leq n \leq N - 1$$ в моем случае два окна на каждую группу, то есть N = 1024.
        \item Затем для каждой группы применить FFT, отнормировать (то есть взять натуральный логарифм) и найти максимум
        \item объединить в вектор максимумы в каждой группе
    \end{enumerate}

    \subsection*{Описание программы}

    Проект состоит из файла генератора mp3 сэмплов и трех директорий:
    \begin{itemize}
        \item data: хранит файлы .mp3 и .pcm
        \item pytests: хранит файл с программой вычисляющей то же, что и дано в задании. Также строит график dB(Hz) после FFT
        \item src: хранит 5 файлов:
            \begin{itemize}
                \item decoder.hpp: хранит заголовок для функции декодировщика mp3 файла в pcm
                \item decoder.cpp: хранит реализацию функции декодировщика mp3 файла в pcm
                \item fft.hpp: хранит заголовки для функций FFT и Hann\_Window 
                \item fft.cpp: хранит реализации функций FFT и Hann\_Window 
                \item main.cpp: драйвер
            \end{itemize}
    \end{itemize}

    \subsection*{Дневник отладки}

    \begin{enumerate}
        \item 20.12.22 Изучена теория
        \item 22.12.22 Написаны тесты
        \item 25.12.22 Написан код
        \item 26.12.22 Протестирован код
    \end{enumerate}

    \subsection*{Тест производительности}

    \begin{tabular}{ | l | l | }
        \hline
            Кол-во элементов в pcm & Время (в мкс) \\ \hline
            36864  & 125777 \\
            73728  & 134815 \\
            73728  & 144147 \\
            110592 & 243023 \\
            147456 & 307250 \\
            184320 & 289367 \\
            258048 & 407468 \\
            294912 & 445275 \\
            331776 & 504473 \\
            368640 & 560305 \\
            405504 & 610382 \\
            2174976 & 3122081 \\
            3207168 & 4875536 \\
        \hline
    \end{tabular}
 \\

\begin{tikzpicture}
    \begin{axis} [
        legend pos = north west,
        ymin = 0
    ]
    \legend{
        FFT, 
        n*log(n)
    };
    \addplot coordinates {
        (36864,125777) (73728,134815) (73728,144147) (110592,243023) (147456,307250) (184320,289367) (258048,407468) (294912,445275) (331776,504473) (368640,560305) (405504,610382)
    };
    \addplot coordinates {
        (36864,387624) (73728,826353) (73728,826353) (110592,557795) (147456,1754915) (184320,2234774) (258048,3215510) (294912,3714249) 

    };

    \end{axis}
\end{tikzpicture}
    
    Как видно график nlogn ограничивает наш график севрху, поэтому сложность O(n*log(n))

    \subsection*{Выводы}

    Проделав данный курсовой проект я узнал много нового про обработку звука. Понял как работает mp3 и как его преобразовывать в вектор амплитуд. Также разобрался и написал код быстрого преобразования фурье. Протестировал программу на реальных данных и сравнил результаты с работой программы, написанной на питоне с применением функции fft из модуля scipy. Данный проект будет хорошей основой для написания приложения для аудиопоиска. 

    \end{document}
