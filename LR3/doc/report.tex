\documentclass[12pt]{article}

\usepackage{fullpage}
\usepackage{multicol,multirow}
\usepackage{tabularx}
\usepackage{ulem}
\usepackage[utf8]{inputenc}
\usepackage[russian]{babel}
\usepackage{listings}
\usepackage[T2A]{fontenc}

\begin{document}

\section*{Лабораторная работа №\,3 по курсу дискрeтного анализа: Исследование качества программ}

Выполнил студент группы 8О-207Б-20 МАИ \textit{Чекменёв Вячеслав Алексеевич}.

\subsection*{Условие}

Общая постановка задачи:
\begin{enumerate}
\item Исследовать программу на наличие утечек памяти с помощью утилиты valgrind.

\item Исследовать скорость выполнения программы с помощью утилиты gprof, выявить недочеты производительности.

\end{enumerate}

\subsection*{Дневник выполнения работы}

\subsection{Valgrind -- memory}
Проанализируем нашу программу на наличие утечек памяти с помощью valgrind. \\
Запустим программу: \\
\texttt{valgrind --tool=memcheck --leak-check=full --show-leak-kinds=all 
--leak-resolution=med ./main < test.txt} \\\\
Возьмем тест на 10000 строк с функциями Save, Load, AddNode, DeleteNode, ReturnNode\\
Проанализируем полученный отчет: \\
\begin{lstlisting}
==15100== LEAK SUMMARY:
==15100==    definitely lost: 0 bytes in 0 blocks
==15100==    indirectly lost: 0 bytes in 0 blocks
==15100==      possibly lost: 0 bytes in 0 blocks
==15100==    still reachable: 122,880 bytes in 6 blocks
==15100==         suppressed: 0 bytes in 0 blocks
==15100== 
==15100== ERROR SUMMARY: 0 errors from 0 contexts (suppressed: 0 from 0)
\end{lstlisting}
Категория "still reachable" в отчете об утечке Valgrind говорит о том, что эти блоки не были освобождены, но они могли бы быть освобождены, потому что программа все еще отслеживала указатели на эти блоки памяти.\\


Эту ошибку можно исправить убрав три строчки кода:

\begin{lstlisting}
std::ios::sync_with_stdio(false);
std::cin.tie(0);
std::cout.tie(0); 
\end{lstlisting}

Теперь посмотрим, что нам выдает Valgrind:
\begin{lstlisting}
==3703== HEAP SUMMARY:
==3703==   total heap usage: 23,235,515 allocs, 23,235,515 frees, 774,349,106 bytes allocated
==3703== LEAK SUMMARY:
==3703==    definitely lost: 0 bytes in 0 blocks
==3703==    indirectly lost: 0 bytes in 0 blocks
==3703==      possibly lost: 0 bytes in 0 blocks
==3703==    still reachable: 0 bytes in 0 blocks
==3703==         suppressed: 0 bytes in 0 blocks
==3703== 
==3703== ERROR SUMMARY: 0 errors from 0 contexts (suppressed: 0 from 0)
\end{lstlisting}

\subsubsection{gprof -- time}

Исследуем нашу программу с помощью gprof. \\
Для использования профилятора следует добавить флаг -pg в команды компиляции. После этого можно запускать программу обычным образом.
Возьмем тест на 25000 строк с функциями Save, Load, AddNode, DeleteNode, ReturnNode\\
После завершения нашей программы будет сгенерирован файл отчета \textbf{gmon.out}, запустим утилиту gprof для получения анализа: \\
\texttt{gprof main.o > gprof.txt} \\\\
Проанализируем полученный отчет: \\
\begin{table}[ht]
\centering
\begin{tabular}{p{0.1\linewidth} | p{0.2\linewidth} | p{0.3\linewidth}}
\hline
seconds-self & calls & name  \\ \hline
   2.83    &   156591265 &   CharToBinary                     \\ \hline
   0.91    &   158215535 &   RetrieveBit                      \\ \hline
   0.30    &     4459264 &   ReturnNodePrivate       \\ \hline
   0.00    &        6419 &   ReturnNode              \\ \hline
   0.21    &     4446338  &  AddNodePrivate          \\ \hline 
   0.09    &     4446558  &  AddNode                 \\ \hline
   0.00    &       6287   &  DeleteNode              \\ \hline 
   0.04    &       2993   &  Load                    \\ \hline     
   0.01    &       3056   &  Save                    \\ \hline 
   0.03    &       3055   &  FillNodesVectorPrivate  \\ \hline
   0.00    &       3056   &  FillNodesVector         \\ \hline     
   0.04    &       2993   &  ClearNode               \\ \hline     
\end{tabular}
\end{table}

Как мы видим, много времени тратится на функцию ReturnNodePrivate, это из-за того, что здесь при каждом рекурсивном вызове функции ReturnNodePrivate проверяется header на NULL, этого можно делать один раз.

\begin{lstlisting}
    if (header == NULL) {
        return NULL;
    }
\end{lstlisting}

Также много времени тратится на функции CharToBinary, RetrieveBit, в них можно было использовать побитовые операции вместо циклов, однако я этого не сделал

\subsection*{Найденные недочеты}
\begin{table}[ht]
\centering
\begin{tabular}{p{0.4\linewidth} | p{0.4\linewidth}}
\hline
Проблема & Инструмент \\ \hline
Утечка памяти "still reachable" & valgrind \\ \hline
\end{tabular}
\end{table}

\subsection*{Выводы}
После использования таких инструментов как gprof и valgrind стало ясно что они действительно способны указать на проблемные места в программе. С помощью gprof было найдено вычислено время выполнения каждой функции в программе. Инструмент valgrind помог найти утечки памяти в программе.

\end{document}

